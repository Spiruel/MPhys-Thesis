\chapter{Method}

\lipsum[100]


\section{N-body Simulations}

\lipsum[100]


%%%%%%%%%%%%%%%%%%%%%%%%%%%%%%%%%%
%%  Beispiel fuer eine Tabelle  %%
%%%%%%%%%%%%%%%%%%%%%%%%%%%%%%%%%%

\begin{table}[htb]
\centering
\begin{tabular}{l|l}
Erste Spalte & Zweite Spalte \\ \hline
Eintrag & Eintrag
\end{tabular}
 \caption[Kurzform f"ur das Tabellenverzeichnis]{Dies ist die Erkl"arung zur Tabelle.}
\end{table}

\section{Model}

Model employed this basic framework:

\begin{itemize}
    \item isotropic massless test particles
\end{itemize}
\section{Simulations}

\section{Backwards-step simulation of solar system}

\section{Fragmentation}

\section{Loss cone dynamics}

ratio of prograde to retrograde comets
clustering of aphelion directions
exotic particles
\cite{1997astro.ph..5153W}