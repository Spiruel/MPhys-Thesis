\documentclass[12pt]{book}

\usepackage{a4wide}
%\usepackage{lastpage}
\usepackage{fancyhdr}
\usepackage{graphicx}
%\usepackage{pgfplots}
\graphicspath{{figures/}}

\usepackage[font=small]{caption}
\setlength{\abovecaptionskip}{-1ex}
\setlength{\belowcaptionskip}{-3ex}

\usepackage[UKenglish]{babel}% http://ctan.org/pkg/babel
\usepackage[UKenglish]{isodate}% http://ctan.org/pkg/isodate
\usepackage{lipsum}
\usepackage{amsmath}
\usepackage{mathabx}
\usepackage[top=1in,bottom=1in,right=1in,left=1in,head=1in]{geometry}

\usepackage{multirow}
\usepackage{longtable}
%\usepackage[table,xcdraw]{xcolor}
\usepackage{booktabs}
\usepackage{rotating}

\usepackage{subcaption}

\usepackage[breaklinks,colorlinks,
   urlcolor=black,citecolor=black,linkcolor=black]{hyperref}
   
%\usepackage{xcolor}
%\definecolor{xlinkcolor}{cmyk}{1,1,0,0} % FOR BLUE CITE LINKS
%\hypersetup{citecolor=xlinkcolor}
            
\usepackage[authoryear, round, sort]{natbib}
\setlength{\bibsep}{.25ex plus 0ex} %bibliography linespacing

\usepackage{amsmath}
\usepackage{xstring}
\usepackage{tocloft}
\usepackage{blindtext}

\usepackage{aas_macros}
%\let\cite\citep

\pagestyle{fancyplain}
\fancyhf{}
\rfoot{\thepage}

%\renewcommand{\chaptermark}[1]%
%         {\markboth{\thechapter.\ #1}{}}
%\renewcommand{\sectionmark}[1]%
%         {\markright{\thesection\ #1}}
%\lhead[\fancyplain{}{\bfseries\thepage}]%
%    {\fancyplain{}{\bfseries\rightmark}}
%\rhead[\fancyplain{}{\bfseries\leftmark}]%
%    {\fancyplain{}{\bfseries\thepage}}
\cfoot{}

\usepackage{titlesec}
\titlespacing\section{0pt}{4pt plus 4pt minus 2pt}{4pt plus 2pt minus 2pt}
\titlespacing\subsection{0pt}{4pt plus 4pt minus 2pt}{4pt plus 2pt minus 2pt}
\titlespacing\subsubsection{0pt}{4pt plus 4pt minus 2pt}{4pt plus 2pt minus 2pt}

\titleformat{\chapter}[hang]
    {\normalfont\huge\bfseries}{\thechapter}{20pt}{\Huge}
\titlespacing*{\chapter}{0pt}{0pt}{20pt}

\makeatletter

\providecommand\phantomsection{}% for hyperref

\newcommand\listofillustrations{%
    \chapter*{List of Tables \& Figures}%
    \phantomsection
    \section*{Figures}%
    \phantomsection
    \@starttoc{lof}%
    \bigskip
    \section*{Tables}%
    \phantomsection
    \@starttoc{lot}}

\makeatother

\begin{document}

{\let\cleardoublepage\clearpage 

\titlepage

  %\frontmatter

{
  \thispagestyle{empty}
  \vspace*{\stretch{1}}
  {\parindent0cm
   \rule{\linewidth}{.7ex}}
  \begin{flushright}

    \vspace*{\stretch{1}}
    \sffamily\bfseries\Huge
    Doomsday: Cometary Remnant Impacts\\and their Mitigation\\
    \vspace*{\stretch{1}}
    \sffamily\bfseries\large
    Samuel Bancroft
    \vspace*{\stretch{1}}
  \end{flushright}
  \rule{\linewidth}{.7ex}
  
\vspace{3ex}
\small{
\begin{center}%
  {\bfseries \vspace{-.5em}{Abstract}}%
\end{center}%
\vspace{-3ex}
\noindent \quotation{Recent developments in Solar System dynamics have indicated that the comet impact hazard is of a much greater complexity than first realised. It is likely that previous comet impacts have provided some of the most globally destructive events in our planet's history and that, while inefficient impactors, cometary material currently exists in the Solar System that poses a direct threat to Earth. In order to investigate the comet threat, a large scale N-body simulation is performed backwards in time for 10 Myr in order to examine the various dynamical pathways for a cometary population scattered in the Solar System to migrate onto Earth-crossing orbits. We determine analytically that a host of dynamical arguments exist for the delivery of numerous different classes of cometary material towards Earth. We find that comets spend little of their dynamical lifetimes in regions of the Solar System visible to Earth. We investigate loss cone dynamics as a useful tool to monitor rapidly evolving cometary material. A decision tree model is presented as a possible mitigation framework to help identify hazardous cometary material in the inner Solar System. The predictive model identifies many real objects evolving similarly to those selected from our N-body simulation. We conclude with an assessment that current planetary defence efforts are misunderstanding the relevance of cometary remnant impacts. The focus on short-period objects such as asteroids to the almost complete neglect of unobservable long-period impactors is deemed injudicious, and a greater awareness is suggested in order to correct this imbalance.}
}%Context, Aims, Methods, Results, Conclusion 

  \vspace*{\stretch{5}}
  \begin{center}
    \includegraphics[width=2in]{DU_2-col_sml.pdf}
  \end{center}
  \vspace*{\stretch{1}}
  \begin{center}\sffamily\LARGE{11th April 2018}\end{center}

  \thispagestyle{empty}

  \vspace*{\stretch{3}}
  \begin{center}
  \large \emph{Supervisor: Dr Richard Wilman}\\
  \vspace{5ex}
  \emph{Submitted in partial satisfaction of the requirements for the degree
F301 ``Physics (4 Years)'' at Durham University}
  \end{center}
}

\endtitlepage

  \pagenumbering{gobble} %turn off page numbers until intro

  \tableofcontents
  %\listofillustrations
  %\listoffigures
  %\listoftables
   
  \mainmatter\setcounter{page}{3}
  %\widowpenalties 1 10000 %penalty of 10000 (no break) between all lines in every para, prohibiting mid-paragraph page breaks
  \raggedbottom
  \chapter{Introduction}

\lipsum[100]
\cite{2014MNRAS.437L..71S}.


\section{Motivation}

\lipsum[100]


%%%%%%%%%%%%%%%%%%%%%%%%%%%%%%
%%  Einbinden einer Grafik  %%
%%%%%%%%%%%%%%%%%%%%%%%%%%%%%%

\begin{figure}[htb]
  \centering
  \includegraphics[scale=0.5]{DU_2-col_sml.pdf}
  \caption[Kurzform f"ur das Abbildungsverzeichnis]{Dies ist die Erkl"arung zum Bild.}
\end{figure}


\section{Comet Origins}

\lipsum[100] \cite{FERNANDEZ198746}.

\lipsum


  \chapter{Method}

\lipsum[100]


\section{N-body Simulations}

\lipsum[100]


%%%%%%%%%%%%%%%%%%%%%%%%%%%%%%%%%%
%%  Beispiel fuer eine Tabelle  %%
%%%%%%%%%%%%%%%%%%%%%%%%%%%%%%%%%%

\begin{table}[htb]
\centering
\begin{tabular}{l|l}
Erste Spalte & Zweite Spalte \\ \hline
Eintrag & Eintrag
\end{tabular}
 \caption[Kurzform f"ur das Tabellenverzeichnis]{Dies ist die Erkl"arung zur Tabelle.}
\end{table}

\section{Model}

Model employed this basic framework:

\begin{itemize}
    \item isotropic massless test particles
\end{itemize}
\section{Simulations}

\section{Backwards-step simulation of solar system}

\section{Fragmentation}

\section{Loss cone dynamics}

ratio of prograde to retrograde comets
clustering of aphelion directions
exotic particles
\cite{1997astro.ph..5153W}
  \chapter{Results \& Discussion}
\label{chap:results}

\section{Cometary evolution in the inner Solar System}
%We operationally define τ stream as the time taken for the median D-parameter of any two test meteoroids to grow beyond a given threshold. The D-parameter was originally introduced by Southworth & Hawkins (1963) for meteor shower identification; it is essentially a measure of the similarity between a pair of orbits denoted as A and B:

\begin{equation}
\begin{split}
    D_{A,B}^2 = (q_B - q_A)^2 + (e_B - e_A)^2 + \left(2\sin{\dfrac{I}{2}}\right)^2
    \\ + \left[(e_A + e_B)\sin{\dfrac{\Pi}{2}}\right]^2 ~,
\end{split}
\end{equation}

where,

\begin{equation}
\begin{split}
    I = \arccos[{\cos{i_A}\cos{i_B}+\sin{i_A}\sin{i_B}\cos{\Omega_A-\Omega_B}}]~,\\
    \Pi = \omega_A - \omega_B + 2\arcsin{\left(\cos{\dfrac{i_A + i_B}{2}\sin{\dfrac{\Omega_A - \Omega_B}{2}\sec{\dfrac{I}{2}}}}{}\right)}~.
\end{split}
\end{equation}

\subsection{Dynamical lifetimes}

\subsection{Loss Cone Dynamics}

\begin{equation}
    J^2 = [\mu(2q_* - (q_*^2/a))] \approx 2q_*\mu
\end{equation}

\subsection{Random Walks in energy space}

\subsection{Resonance Hopping}

%\section{Fragmentation}

%\section{Loss cone dynamics}

%ratio of prograde to retrograde comets
%clustering of aphelion directions
%exotic particles
%random walk and resonance hopping
%\cite{1997astro.ph..5153W}

\section{Worst-case scenarios}
  \chapter{Mitigation Framework}
\label{chap:mitigation}

In the event of a comet remnant impact, the impacting material is likely to often pass perihelion (and thus approach impact with the Earth) on the order of a few years after discovery. In the most extreme examples, highly eccentric LPC nuclei may only become visible once within the orbit of Jupiter - leaving only a few month's time to react. 

Therefore, strategies that mitigate against Earth-bound cometary material are of a different nature to the more popular proposed planetary defence efforts that focus primarily on asteroids. Such strategies focus on mitigating against high energy objects with a short-warning time. Recent discussions of direct comet mitigation in literature range from the use of large phased-array lasers illuminating LPC nuclei from afar \citep{2016PASP..128d5001Z}, to a nuclear explosive device affixed to an interceptor craft that waits in stand-by in an orbit around Earth \citep{7500925, HUSSEIN2016488}. %A nuclear detonation is one of the few plausible mitigation strategies that could sufficiently deflect a comet away from Earth, and even then may only be feasible if sent to meet the comet almost immediately upon discovery. 

However in this paper we advocate a more detection-based approach, developing a Random Forest classification model to detect hazardous evolving cometary material scattered in the Solar System from our current catalogues of small bodies detected from Earth. Such a framework may permit greater understanding of the nature of hazardous cometary material presently being scattered through the Solar System, and therefore inform us on how best to develop a response.

We develop a mitigation framework that acts purely on the dynamical qualities of a comet's orbit, and neglects features such as a comet's size and other physical attributes. In developing a detection method to look for evidence of dangerous cometary material, regardless of size, composition etc - we hope to look for cometary remnants that could themselves be an direct threat to Earth, or that could be genetically associated with an as of yet unobserved threatening parent comet nuclei or host of fragmented material.

\section{Random Forest Model}

As discovered in \S~\ref{chap:results}, cometary material in the Solar System is continuously scattered on through the Solar System, along vibrant dynamical pathways dominated by resonances, diffusion, and chaos. While N-body integrations help us understand the various dynamical pathways that a comet may take, they are inherently computationally intensive processes that make them unsatisfactory for examining the long-term evolution of a specific object and how it may pose a risk to our planet.

We instead select a large sample of orbital features from our N-body simulation, and use this to train a predictive model in order classify objects with similar orbital characteristics as hazardous. Hazardous orbits from the simulation are used in a learning algorithm alongside non-hazardous orbits with more asteroidal dynamical characteristics - sourced from the JPL small body database.

We selected a large number of datapoints from our simulation data at various points in time. In particular we selected simulated comets that i) entered the Earth's loss cone only once ii) most of their dynamical lifetimes beyond the orbit of Jupiter iii) were initially bound to the Solar System. This selection was chosen in order to prioritise cometary material that could represent the dominant impact hazard, ie. those that rapidly and chaotically evolve through the Solar System, and arrive in the NEO environment with short warning times. From this selection of datapoints we used an 8:2 split for training and validation purposes respectively.

The Random Forest Classification Model is made up of an ensemble of many classification decision trees, which themselves individually separate the feature space of a training data set into smaller and smaller partitions. In an attempt to limit a decision tree's ability to `overfit' to the data it's been trained on, we aggregate a number of decision trees into one final structure known as a random forest. 

We implement our random forest using the \textsc{CART} decision tree algorithm \citep{breiman2017classification} from the \texttt{Python} library \texttt{scikit-learn} \citep{cournapeau2015sci}. During the training process, at the point of a particular feature partition, the algorithm maximises the efficiency of a tree such that the Gini impurity $I_G$ of the two resulting feature partitions is minimised. The Gini impurity can be calculated by summing the probability $p_f$ of an item with label $f$  being chosen multiplied by the probability of a mistake in categorising that item.  It is given by,

\vspace{-1ex}
\begin{equation}
   I_G =  \sum_{k = 0}^k p_f(1-p_f)~,
\end{equation}

where $K$ is the number of classes. For our model we use $K=2$, where the two classes denote hazardous cometary material and non-hazardous material. 

We trained our random forest algorithm using five orbital elements as features. These were $(a,e,i,\Omega,\omega)$ - semi-major axis, eccentricity, inclination, longitude of ascending node, and argument of perihelion respectively. These features completely describe the shape and orientation of a comet's orbit relative to the Sun, neglecting the comet's particular position in time. These features therefore encode all the necessary information (for example velocity, value of $T_J$) that permit a classification on whether the object is hazardous or not.

100 individual trees were trained with a maximum depth of 10 levels in order to reduce overfitting. Once all the individual decision trees had been calculated, with each tree operating on a random subset of the training data, the result was then averaged into a random forest - much more accurate than an individual decision tree. With a trained random forest, orbital elements were then be propagated through the structure and a predicted classification was then output.

\section{Results}

Our model performed well against our validation data, with 77\% of results predicting a hazardous orbit being correct, with 85\% of the hazardous comets classified successfully. 

%Unlike other predictive models which have more of a black box nature, each trained decision tree in our random forest could be individually studied and interpreted by humans. By examining the total reduction of $I_G$ due to a particular feature across all our decision trees, we were able to calculate the relative importance of each feature used for classification. We found that $a$ and $e$ were the most important determining features of an object's orbit as to whether it is hazardous or not.

When the random forest was applied to all known small bodies as catalogued by the JPL small body database, around 192 ($\sim0.1\%$) of the objects were classed as having similar orbital evolutionary characteristics as the dangerous comets in our simulation. This group of objects were made up 70\% LPCs, 19\% SPCs, 11\% eccentric asteroids and minor planets - all at various stages in their dynamical evolution. Notable classifications included; the NEO 4341 Poseidon - associated with the Taurid Complex of meteor showers \citep{2001A&A...373..329B}, 944 Hidalgo - one of the first centaurs to be discovered, and 109P/Swift-Tuttle - described by \cite{verschuur1997impact} as the `the single most dangerous object known to humanity'.


%Notable classifications include objects on eccentric orbits that are already classified as NEOs and are therefore relevant to the human epoch. These included the NEOs 2003 EH1, 2004 UL, and 2004 TG$_{10}$. The notable 109P/Swift–Tuttle comet was classified as hazardous.

%https://www.repository.cam.ac.uk/bitstream/handle/1810/246615/MNRAS-2015-Shannon-2059-64.pdf?sequence=1 read S5
  \chapter{Conclusion}
\label{chap:conclusion}

\lipsum[100]

%Directed Energy NEO Deflection Simulations
%Long period comets passing through the inner solar system often pass perihelion (and thus the Earth) in under 2 yr after discovery. Such short notice coupled with the highly eccentric and inclined orbits of many of these comets makes the deliveryof a stand-on system to a threatening comet infeasible. ZHANG 2016

humans wired to perceive short term threats

mitigation framework based purely on the dynamical qualities of orbits, and less abut lightcurves etc as we don't have that characterised. Any flags in the framework that says comets, but are quite small objects at the very least identify cometary material that could be genetically associated with a parent nuclei.

  \backmatter
  \small %\footnotesize
\linespread{.95}\selectfont

\addcontentsline{toc}{chapter}{Bibliography}

\bibliographystyle{aasjournal-hyperref}
\bibliography{references}

\linespread{1}\selectfont
  \markboth{}{}
  
  \appendix

\renewcommand{\thefigure}{A.\arabic{figure}}
\setcounter{figure}{0}
\renewcommand{\thetable}{A.\arabic{table}}
\setcounter{table}{0}

\chapter{Appendix}
\section*{}

\begin{figure}[h!]
    \centering
    \includegraphics{losscone.png}
    \caption[Catalogued asteroids and comets in Earth's loss cone]{Semi-major axes vs eccentricities of observed asteroids (blue), comets (red) and potentially hazardous asteroids (yellow). Data sourced from the IAU MPC. Meteorites with pre-impact orbits determined by \cite{doi:10.1093/mnras/stv378} are plotted as green circles. The Earth aphelion and perihelion are plotted as dashed lines.}
    \label{fig:loss_cone}
\end{figure}

\begin{figure}[h!]
    \centering
    \includegraphics{figures/error.pdf}
    \caption[Simulation error]{Relative energy error over time. Black line shows the moving average of all the simulation runs.}
    \label{fig:error}
\end{figure}

%\section{Worst-case scenarios}

\begin{figure}[h!]
    \centering
    \includegraphics{radiants.pdf}
    \caption[Distribution of apparent radiants]{Distribution of apparent radiants for comets at the point of impact. Geocentric equatorial coordinates (J200) plotted for various comet threats. Green line and boundaries represent the plane of the ecliptic $\pm5^\degree$, black line and boundaries represent the galactic plane $\pm10^\degree$. The galactic centre is plotted as black circle. A 2D kernel density estimate is shown.}
    \label{fig:radiants}
\end{figure}

\begin{figure}[h!]
    \centering
    \includegraphics[width=\textwidth]{tree1.pdf}
    \vspace{.5ex}
    \caption[Example decision tree]{Example decision tree used in the random forest, expanded for the first few layers. Samples represent the percentage of training datapoints associated with a particular node. Node fill colour denotes the majority class for classification.}
    \label{fig:tree}
\end{figure}

\begin{figure}[h!]
    \centering
    \includegraphics[width=\textwidth]{gini.pdf}
    \caption[Relative importance of orbital characteristics in learning algorithm]{Relative importance of each feature used for classification in the random forest learning algorithm. Gini importance calculated by examining the total reduction of $I_G$ due to a particular feature across all our decision trees.}
    \label{fig:gini}
\end{figure}


\begin{sidewaystable}
\caption[Initial conditions]{Initial conditions of the solar system as of 01-12-2017 12:00 UTC. Data sourced from NASA JPL Horizons database.}\vspace{3ex}
\label{table:init_con}
\begin{tabular}{llllllll} \toprule \toprule
\multicolumn{1}{c}{Object} & \multicolumn{1}{c}{Mass / $M_{\odot}$} & \multicolumn{1}{c}{$x$ / AU} & \multicolumn{1}{c}{$y$ / AU} & \multicolumn{1}{c}{$z$ / AU} & \multicolumn{1}{c}{$v_x$ / AU$\;$yr$^{-1}$}& \multicolumn{1}{c}{$v_y$  / AU$\;$yr$^{-1}$}& \multicolumn{1}{c}{$v_z$  / AU$\;$yr$^{-1}$}\\ \midrule
Sun & 1.000E+00 & 1.977E-03 & 5.974E-03 & -1.244E-04 & -2.047E-03 & 1.909E-03 & 4.906E-05 \\
Mercury & 1.660E-07 & 3.336E-01 & 8.143E-02 & -2.438E-02 & -4.270E+00 & 1.048E+01 & 1.248E+00 \\
Venus & 2.448E-06 & -4.901E-01 & -5.244E-01 & 2.100E-02 & 5.361E+00 & -5.058E+00 & -3.788E-01 \\
Earth-Moon system & 3.040E-06 & 3.515E-01 & 9.280E-01 & -1.647E-04 & -5.980E+00 & 2.206E+00 & -1.903E-05 \\
Mars & 3.227E-07 & -1.649E+00 & 3.172E-03 & 4.033E-02 & 1.975E-01 & -4.673E+00 & -1.028E-01 \\
Jupiter & 9.548E-04 & -4.396E+00 & -3.189E+00 & 1.115E-01 & 1.586E+00 & -2.100E+00 & -2.676E-02 \\
Saturn & 2.859E-04 & -1.129E-01 & -1.006E+01 & 1.793E-01 & 1.926E+00 & -2.944E-02 & -7.613E-02 \\
Uranus & 4.366E-05 & 1.778E+01 & 8.962E+00 & -1.970E-01 & -6.572E-01 & 1.216E+00 & 1.303E-02 \\
Neptune & 5.151E-05 & 2.865E+01 & -8.684E+00 & -4.815E-01 & 3.249E-01 & 1.104E+00 & -3.022E-02 \\
\bottomrule
\end{tabular}
\end{sidewaystable}

\section*{Code}

All the \texttt{Python} code used throughout the course of this investigation can be viewed online at \url{https://github.com/Spiruel/L4_Project}.


%UNCOMMENT TO ALLOW ACKNOWLEDGEMENTS
 % \addcontentsline{toc}{chapter}{\protect Acknowledgements}


\chapter*{Acknowledgements}

Thanks to

} %from \let\cleardoublepage\clearpage 

\end{document}
