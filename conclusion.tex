\chapter{Conclusion}
\label{chap:conclusion}

\lipsum[100]
\lipsum[100]
\lipsum[100]
\lipsum[100]
\lipsum[100]

%Directed Energy NEO Deflection Simulations
%Long period comets passing through the inner solar system often pass perihelion (and thus the Earth) in under 2 yr after discovery. Such short notice coupled with the highly eccentric and inclined orbits of many of these comets makes the deliveryof a stand-on system to a threatening comet infeasible. ZHANG 2016

%Our understanding of the dynamics of the Kuiper Belt and Oort Cloud is at a very early stage, but what’s the betting that these turn out to be vibrant dynamic structures dominated by resonances, gaps, and chaos, 

The large orbital period and isotropic inclinations make LPCs rare and inefficient Earth impactors, but little is known about the nature of the transport mechanisms that bring them close to Earth.

humans wired to perceive short term threats

mitigation framework based purely on the dynamical qualities of orbits, and less abut lightcurves etc as we don't have that characterised. Any flags in the framework that says comets, but are quite small objects at the very least identify cometary material that could be genetically associated with a parent nuclei.