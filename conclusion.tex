\chapter{Conclusion}
\label{chap:conclusion}

We have performed a large scale N-body simulation of cometary dynamics in the Solar System, in an effort to understand the nature of the transport mechanisms that bring impacting comets to Earth. By examining cometary material being continuously scattered through the Solar System, we report dynamical lifetimes in line with previous investigations. 

We discover that impacting comets chart vibrant dynamical pathways towards Earth, with behaviours dominated by resonances, diffusion, and chaos. Such behaviours largely define particular dynamical classes of objects, such as the Centaurs travelling rapid and chaotic routes towards the inner planetary environment. We consider the existence of Encke-type comets masquerading as asteroids in the NEO environment. In the absence of physical decay laws in our simulation model we conclude that such comets are not liable to large fluctuations from comet showers, and that if such a population does exist it may consist of extreme low albedo, long-lived objects that have not yet been detected from Earth.

Throughout our analysis we find that retrograde comets display a propensity to persist and be scattered less in the planetary environment. From this we conclude that dangerous retrograde comets may be more likely to reach Earth, but at the expense of more physical ageing due to a slower orbital evolution.

We conclude that the effects of the giant planets act as a significant protection against comets reaching Earth, with objects of large semi-major axes being particularly efficient at circumventing the barrier effect. This has implications for the prospect of very large comet showers circumventing the barrier and reaching into the NEO environment on short timescales. We find that analysis of planetary loss cones can be a useful tool in differentiating rapidly evolving showers of comets with short warning times from objects of a more asteroidal dynamical nature.

By exploring the use of a decision tree predictive model, we provide an interpretable framework that can be used to identify hazardous cometary material. From the model results we find that while large orbital period and isotropic inclinations make LPCs rare and inefficient Earth impactors, comets destined for a collision with our planet may well already exist in our Solar System. Such a conclusion is based purely on dynamical grounds - and future work involving more explicit modelling of cometary physical processes is encouraged in order to assess any additional constraints that these may introduce.

Our understanding of the geological history of our planet in the context of impact hypotheses is at an early stage, and only more work will reveal whether more exceptionally devastating impacts than we may realise have occurred in the past.

We finish by saying that current planetary defence policy may be fallaciously assuming that comet impacts represent a practically negligible impact hazard, whereas simulations suggest that such an event may be more accurately described as one of a more unknown probability. NEO surveys are currently concentrating on perceiving slowly evolving threats based on what we can observe, and despite considerable advances in the last few years, do not merit the present complete lack of awareness when it comes to considering the fact that comets can also reach our planet, and with terrible consequences.
