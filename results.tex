\chapter{Results \& Discussion}
\label{chap:results}

\section{Cometary evolution in the inner Solar System}
%We operationally define τ stream as the time taken for the median D-parameter of any two test meteoroids to grow beyond a given threshold. The D-parameter was originally introduced by Southworth & Hawkins (1963) for meteor shower identification; it is essentially a measure of the similarity between a pair of orbits denoted as A and B:

\begin{equation}
\begin{split}
    D_{A,B}^2 = (q_B - q_A)^2 + (e_B - e_A)^2 + \left(2\sin{\dfrac{I}{2}}\right)^2
    \\ + \left[(e_A + e_B)\sin{\dfrac{\Pi}{2}}\right]^2 ~,
\end{split}
\end{equation}

where,

\begin{equation}
\begin{split}
    I = \arccos[{\cos{i_A}\cos{i_B}+\sin{i_A}\sin{i_B}\cos{\Omega_A-\Omega_B}}]~,\\
    \Pi = \omega_A - \omega_B + 2\arcsin{\left(\cos{\dfrac{i_A + i_B}{2}\sin{\dfrac{\Omega_A - \Omega_B}{2}\sec{\dfrac{I}{2}}}}{}\right)}~.
\end{split}
\end{equation}

\subsection{Dynamical lifetimes}

\subsection{Loss Cone Dynamics}

\begin{equation}
    J^2 = [\mu(2q_* - (q_*^2/a))] \approx 2q_*\mu
\end{equation}

\subsection{Random Walks in energy space}

\subsection{Resonance Hopping}

%\section{Fragmentation}

%\section{Loss cone dynamics}

%ratio of prograde to retrograde comets
%clustering of aphelion directions
%exotic particles
%random walk and resonance hopping
%\cite{1997astro.ph..5153W}

\section{Worst-case scenarios}